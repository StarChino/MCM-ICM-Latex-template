%%%%%%%%%%%%%%%%%%%%%%%%%%%%%%%%%%%%%%%%%%%%%%%%%%%%%%%
%%% 美国大学生数学建模竞赛(MCM/ICM)论文模板
%%% 来源网站:www.latexstudio.net
%%       2010 -- 2015 by Zhaoli Wang
%%       2014 -- 2019 by Liam Huang
%%       2019 -- 2026 by latexstudio.net
%%% 中文注释:小嗷犬 blog.marquis.eu.org
%%% 部分修改:StarChino (Feb2026)
%%%%%%%%%%%%%%%%%%%%%%%%%%%%%%%%%%%%%%%%%%%%%%%%%%%%%%%
%%% code: 代码文件夹
%%% figures: 图片文件夹
%%% *.cls: LaTeX 格式文件
%%% *.tex: LaTeX 文档文件
%%% *.bib: Bib 引用文献源文件
%%%%%%%%%%%%%%%%%%%%%%%%%%%%%%%%%%%%%%%%%%%%%%%%%%%%%%%
%%% 可能用到的网站
%%% LaTeX公式编辑器:https://www.latexlive.com/
%%% Diagram流程图绘制:https://www.drawio.com/
%%%%%%%%%%%%%%%%%%%%%%%%%%%%%%%%%%%%%%%%%%%%%%%%%%%%%%%
%%% 模板参数设置
\documentclass{mcmthesis}  % 文档类型
\mcmsetup{CTeX = false,   % 使用 CTeX 套装时,设置为 true
        tcn = 1111111,   % 队伍控制号(记得修改)
        problem = ABCDEF,  % 选题(记得修改)
        sheet = true,   % sheet页
        titleinsheet = true,   % sheet页显示标题
        keywordsinsheet = true,  % sheet页显示关键词
        titlepage = false,   % 标题页(没见过要用)
        abstract = true  % 摘要
        }
%%%%%%%%%%%%%%%%%%%%%%%%%%%%%%%%%%%%%%%%%%%%%%%%%%%%%%%
%%% 导入宏包和引用文献源
\usepackage{palatino}  % 帕拉提诺体字体宏包
\usepackage{lipsum}  % 导入生成段落的宏包
\usepackage[hyperref=true,style=ieee]{biblatex}  % biblatex参考文献宏包
\usepackage{makecell}%表格内换行
\usepackage{longtable}%长表格
\usepackage{setspace}
\usepackage{eso-pic}
\usepackage{graphicx}
\usepackage{tcolorbox}
\usepackage[svgnames]{xcolor} % 提供丰富的颜色名称
\addbibresource{ref.bib}  % 添加引用文献bib源
%%%%%%%%%%%%%%%%%%%%%%%%%%%%%%%%%%%%%%%%%%%%%%%%%%%%%%%
%%% 文档信息设置
\title{Title}  % 文章标题(记得修改)
\author{\small Team 1111111}  % 作者,开启标题页才会显示
\date{\today}  % 日期,开启标题页才会显示
%%%%%%%%%%%%%%%%%%%%%%%%%%%%%%%%%%%%%%%%%%%%%%%%%%%%%%%
%%% ===============文档开始===============
\begin{document}  % 文档
\begin{abstract}  % 摘要
\hspace{1em}This paper

\textbf{For Task 1},

\textbf{For Task 2}, 

\textbf{For Task 3},

\textbf{For Task 4},

Ultimately,

%%% ===============摘要正文结束===============
\begin{keywords}  % 关键词(记得修改)
keywords1,keywords2,keywords3
\end{keywords}  % 结束关键词
\end{abstract}  % 结束摘要
\maketitle  % 生成sheet页
\begin{spacing}{0.8}
\tableofcontents  % 生成目录表  
\end{spacing}

\newpage  % 开始新的一页
%%%================正文开始==================
\section{Introduction}
\subsection{Background}

\iffalse%注释开始
\fi%注释结束
%为防止出错 部分示例已经注释
%%%%%%%%注释掉的示例 使用时复制修改即可

\begin{figure}[h]  % 图片
\small
\centering  % 居中
\includegraphics[width=5cm]{example1.jpg}  % 引入图片源 支持jpg png pdf
\caption{example} \label{fig:example}  % 标题与标签
\end{figure}  % 图片结束

%This is Figure \eqref{fig:example}.  % 引用图表

This is a cite\cite{toksoz1974structure}.  % 引用文献

\begin{equation}  % 公式,独占一行、居中,自动编号
E = mc^2 \label{aa}  % 标签
\end{equation}  % 公式结束

\begin{equation}  % 公式,独占一行、居中
\nonumber % 不编号
E = mc^2
\end{equation}  % 公式结束

\[a^2+b^2=c^2
\]%单行公式

This is $a^2+b^2=c^2$%行内公式

\begin{itemize}  % 无序列表
\item This is a item.
\item This is a item.
\end{itemize}  % 无序列表结束

\textit{I love math.}  % 斜体

\textbf{I love math.}  % 粗体

\underline{I love math.}  %下划线

%%%%%%%%%%%%%%%%%%%%%%%% 并排图 %%%%%%%%%%%%%%%%%%%%%%%%
\begin{figure}[h]  % 图片
\centering  % 居中
\begin{minipage}[c]{0.48\textwidth}  % 子页
\centering  % 居中
\includegraphics[width=7cm]{example1.jpg}  % 引入图片源
\caption{example} \label{fig:example}  % 标题与标签
\end{minipage}  % 子页结束
\hspace{0.02\textwidth}
\begin{minipage}[c]{0.48\textwidth}  % 子页
\centering  % 居中
\includegraphics[width=7cm]{example2.jpg}  % 引入图片源
\caption{example} \label{fig:example}  % 标题与标签
\end{minipage}  % 子页结束
\end{figure}  % 图片结束
%%%%%%%%%%%%%%%%%%%%%% 并排图结束 %%%%%%%%%%%%%%%%%%%%%%

%%%%%%%%%%%%%%%%%%%%%%%% 三线表 %%%%%%%%%%%%%%%%%%%%%%%%
\begin{table}[htb]  % 表格
\caption{Caption}  % 标题
\label{tab1}  % 标签
\tabcolsep 42pt % 列间距
\begin{tabular*}{\textwidth}{cccc}  % tabular*环境
\toprule  % 顶线
Title a & Title b & Title c & Title d \\
\midrule  % 中线
Aaa & Bbb & Ccc & Ddd \\
Aaa & Bbb & Ccc & Ddd \\
Aaa & Bbb & Ccc & Ddd \\
\bottomrule  % 底线
\end{tabular*}  % tabular*环境结束
\end{table}  % 表格结束
%%%%%%%%%%%%%%%%%%%%%% 三线表结束 %%%%%%%%%%%%%%%%%%%%%%
%%%%%%%%%%%%%%%%%%%%%%%% 长表格(可换页) %%%%%%%%%%%%%%%%
\begin{longtable}{ccc}
\caption{Table2}\\%标题
\hline
    \makecell{UP\\DOWN}&1111&2222\\%表格内换行
\hline
    11&22&33\\
    11&22&33\\
    11&22&33\\
\hline
\end{longtable}
%%%%%%%%%%%%%%%%%%%%%%%% 长表格结束 %%%%%%%%%%%%%%%%


\subsection{Restatement of the problem}

\subsection{Literature review}

\subsection{Our work}

\section{Assumption and Justification}
\subsection{Assumption}
\begin{itemize}
\item \textbf{1.Assumption1}\\
Your assumption.\\
\item\textbf{2.Assumption2}\\
Your assumption.\\
\item\textbf{3.Assumption3}\\
Your assumption.\\
\end{itemize}
\subsection{Nomenclature}

\section{Task1}

\section{Task2}

\section{Task3}

\section{Task4}

\section{Sensitivity analysis of the model}

\section{Evaluation and Extension}


%%%%%%%%%%%%%%%%%%%%%%% 参考文献 %%%%%%%%%%%%%%%%%%%%%%%
\newpage%页数紧张时不建议参考文献单开一页 建议注释掉
\addcontentsline{toc}{section}{References}%将引用文献显示在目录页
\printbibliography  % 打印引用文献列表
%%%%%%%%%%%%%%%%%%%%%%% Memo/建议书 %%%%%%%%%%%%%%%%%%%%%%%
%建议灵活设置 也可以直接导入有字的图片作为背景
\newpage
\thispagestyle{empty}
% --- 1. 背景图片设置 ---
\AddToShipoutPicture*{
    \AtPageLowerLeft{%
        \includegraphics[width=\paperwidth,height=\paperheight]{background.jpg}}}
% --- 2. 写入目录 ---
\phantomsection
\addcontentsline{toc}{section}{Memo}%这里修改目录页Memo名称
\vspace*{-2cm} 
% --- 3. 透明衬底设置 ---
\begin{center}
\begin{tcolorbox}[
    colback=white, 
    opacityback=0.85, 
    colframe=white, 
    % --- 让框向左右页边距延伸 ---
    width=\textwidth,
    grow to left by=1.5cm,   % 向左边距伸进
    grow to right by=1.5cm,  % 向右边距伸进
    % ------------------------------------
    arc=0pt, %圆角尺寸
    boxrule=0pt, %边框
    standard jigsaw,
    % ---文字离白框边缘的距离---
    left=25pt, 
    right=25pt, 
    top=20pt, 
    bottom=20pt
]
    % --- 标题部分:大尺寸、斜体 (itshape)、不加粗 ---
    {\fontsize{26pt}{30pt}\rmfamily\itshape  Title\\ }
    % --- 称呼部分 ---(不用可以注释掉)
    \vspace{3ex} % 标题与称呼之间的间距
    {\large \rmfamily \noindent Dear Members of the MCM Office,\\}
    \vspace{2ex}
    \hrule height 0.5pt % 极细线,低调优雅
    \vspace{4ex}
    % --- 正文部分 ---
    \rmfamily \large % 加大字号
    \setlength{\parskip}{1.2em} % 段落间距
    \setlength{\parindent}{0pt} % 缩进
Your idea.

    
    \vfill % 自动填满剩余空间,让落款沉到底部
    \noindent
    Sincerely yours, \\
    \textbf{Team \showtcn}

\end{tcolorbox}
\end{center}
\lastmainpage%终止页码计数
\clearpage
\pagestyle{fancy}
%%%%%%%%%%%%%%%%%%%%%%AI工具部分%%%%%%%%%%%%%%%%%%%%%
\newpage
\pagestyle{empty}
\section*{\begin{center}Report on Use of AI Tools\end{center}}
\newcounter{ain}
\newcounter{queryn}
\newcounter{outputn}
\newcommand{\aiss}[2]{
        \stepcounter{ain}
        \setcounter{queryn}{0}
        \setcounter{outputn}{0}
        \subsection*{\theain. #1(#2)}
}
\newcommand{\queryss}{
        \stepcounter{queryn}
        Query\thequeryn :\hspace{1em} 
}
\newcommand{\optss}{
        \stepcounter{outputn}
        Output\theoutputn :\hspace{1em} 
}
%==========开始正文 ============
%命令可以帮助自动计数 也可以选择自行书写 不会影响页码自动计算
\aiss{ChatGPT}{ChantGPT5.2-version20260130}%AI和版本
\queryss Your qusetion\\%问题(不要删除换行符号)
\optss AI's idea\\%回答
\queryss Your qusetion\\%问题
\optss AI's idea\\%回答
\aiss{Google Gemini}{Gemini3.0-Flash}%AI和版本
\queryss Your qusetion\\%问题
\optss AI's idea\\%回答
%%%%%%%%%%%%%%%%%%%%%%%%%%%%%%%%%%%%%%%%%%%%%%%%%%%%%%%
\end{document}  % 文档结束
